\section*{2.3}

\noindent
We are asked to show that for $N$ data points uniformly distributed in a a 
$p$-dimensional unit that the median distance from the center is such that:

\begin{gather*}
d(p, N) = (1 - \frac{1}{2^{\frac{1}{N}}})^{\frac{1}{p}}
\end{gather*}

\noindent
The volume of a $p$-sphere is defined as:

\begin{gather*}
V = \frac{\pi^{\frac{p}{2}}}{\Gamma(\frac{n}{2} + 1)} \times r^{p}
\end{gather*}

\noindent
Where $r$ is the radius. For a unit sphere, $r = 1$.

\vspace{5mm}
\noindent
We now consider how long of a radius we would need for a $p$-dimensional 
sub-sphere to encapsulate $x$ fraction of the volume of our unit sphere.

\begin{gather*}
%----------------------------Line 1---------------------------------------------
V \times x = 
\frac{\pi^{\frac{p}{2}}}{\Gamma(\frac{n}{2} + 1)} \times r^{p} \implies 
\frac{\pi^{\frac{p}{2}}}{\Gamma(\frac{n}{2} + 1)} \times x = 
\frac{\pi^{\frac{p}{2}}}{\Gamma(\frac{n}{2} + 1)} \times r^{p} \implies \\ 
%----------------------------Line 2---------------------------------------------
\implies
r^{p} = x \implies 
r = x^{\frac{1}{p}}
\end{gather*}

\noindent
We can now interpret these fractions of volumes as probabilities. The 
probability that a randomly sampled point is within $x^{\frac{1}{p}}$ of the 
origin is equal to $x$. Then, the probability that it is outside of this 
distance is $1 - x$. Then, the probability of all $N$ points being outside of 
the distance $x^{\frac{1}{p}}$ is $(1 - x)^{N}$. 

\vspace{5mm}
\noindent
We now have a relationship between the volume of a sub-sphere and the 
probability of all $N$ data not lying within that sub-sphere. We also have a 
relationship between the volume of the sub-sphere and the radius/the distance 
from the origin of our original unit sphere. This radius corresponds to the 
distance to the closest point. If we set our probability of all points lying 
outside of the volume of our sub-sphere to $\frac{1}{2}$ (the median), we can 
chain together these relationships to find the median distance to the closest 
data point. First, we find the volume that corresponds to the probability that 
all points our outside of this volume equaling $\frac{1}{2}$:

\begin{gather*}
(1 - x)^{N} = \frac{1}{2} \implies 
1 - x = \frac{1}{2^{\frac{1}{N}}} \implies 
x = (1 - \frac{1}{2^{\frac{1}{N}}})
\end{gather*}

\noindent
Next, we relate this volume to its radius:

\begin{gather*}
r = 
x^{\frac{1}{p}} = 
(1 - \frac{1}{2^{\frac{1}{N}}})^{\frac{1}{p}}
\end{gather*}

\noindent
And we are finished! We have the expression for the median distance from the 
origin to the closest data point, and it is exactly the same as the formula 
we wished to derive.